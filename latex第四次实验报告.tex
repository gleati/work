\documentclass{ctexart}
\pagestyle{empty}
\usepackage[a4paper, margin=1in]{geometry}
\usepackage{listings}
\usepackage{xcolor}
\usepackage{enumitem}
\usepackage{graphicx}
\usepackage{hyperref}

% 设置代码块样式
\lstset{
  numbers=none,
  numberstyle=\tiny,
  numbersep=.1em,
  xleftmargin=0em,
  backgroundcolor=\color{lightgray!50},
  columns=fixed,
  breakindent = 1em,
  language=[LaTeX]TeX,
  breaklines=true,
  basicstyle=\ttfamily,
  tabsize=1,
}

\begin{document}

\title{Python 调试和性能分析技巧及 PyTorch 代码示例报告}
\author{郭路通  23020007032}
\date{}
\maketitle

\section{Python 调试和性能分析技巧}

以下是一些具体的 Python 调试和性能分析技巧及其完整代码示例:

\begin{enumerate}[label=\arabic*.]
\item \textbf{使用打印语句进行调试}:
\begin{lstlisting}
def sum_numbers(a, b):
    print(f"Adding {a} and {b}")
    return a + b

result = sum_numbers(3, 5)
print(result)
\end{lstlisting}
打印中间结果以帮助调试。

\item \textbf{使用 pdb 进行交互式调试}:
\begin{lstlisting}
import pdb
def divide(x, y):
    result = x / y
    return result

def main():
    try:
        a = 10
        b = 0
        print( a, "/", b)
        divide(a, b)
    except ZeroDivisionError:
        print("捕获到异常,正在启动调试器...")
        pdb.post_mortem()

if __name__ == "__main__":
    main()
\end{lstlisting}
设置断点,使用 pdb 进行调试。\\
\includegraphics[width=0.5\textwidth]{屏幕截图 2024-09-16 022810.png}
\item \textbf{计算运行时间}:
\begin{lstlisting}
import time

def time_decorator(func):
    def wrapper(*args, **kwargs):
        start_time = time.time()
        result = func(*args, **kwargs)
        end_time = time.time()
        print(f"{func.__name__} took {end_time - start_time} seconds to run")
        return result
    return wrapper

@time_decorator
def example_function(n):
    sum = 0
    for i in range(n):
        sum += i
    return sum

example_function(100000000)
\end{lstlisting}
测量代码片段的执行时间。\\
\includegraphics[width=0.5\textwidth]{屏幕截图 2024-09-16 023231.png}
\item \textbf{使用 cProfile 进行详细性能分析}:
\begin{lstlisting}
import cProfile

def fib(n):
    if n <= 1:
        return n
    else:
        return fib(n-1) + fib(n-2)

cProfile.run('fib(30)')
\end{lstlisting}
进行详细的性能分析。\\
\includegraphics[width=0.5\textwidth]{屏幕截图 2024-09-16 023409.png}
\item \textbf{使用timeit计算函数的运行时间}:
\begin{lstlisting}
import timeit

def test_function():
    result = 0
    for i in range(1000):
        result += i

# 使用 timeit 测试函数执行时间
execution_time = timeit.timeit(test_function, number=100)

print(f"执行时间: {execution_time} 秒")
\end{lstlisting}
计算函数运行时间\\
\includegraphics[width=0.5\textwidth]{屏幕截图 2024-09-16 024416.png}
\end{enumerate}

\section{Python 元编程示例}

以下是一些具体的 Python 元编程示例及其完整代码示例:

\begin{enumerate}[label=\arabic*.]
\item \textbf{使用类装饰器动态修改类}:
\begin{lstlisting}
def add_greeting(cls):
    def say_hello(self):
        print("Hello!")

    cls.say_hello = say_hello
    return cls

@add_greeting
class MyClass:
    pass

obj = MyClass()
obj.say_hello()
\end{lstlisting}
动态为类添加方法。\\
\includegraphics[width=0.5\textwidth]{屏幕截图 2024-09-16 025242.png}
\item \textbf{使用元类动态修改类}:
\begin{lstlisting}
class Meta(type):
    def __new__(cls, name, bases, dct):
        dct['class_level'] = 'modified by metaclass'
        return super().__new__(cls, name, bases, dct)

class MyClass(metaclass=Meta):
    pass

print(MyClass.class_level)
\end{lstlisting}
在类创建时修改类属性。\\
\includegraphics[width=0.5\textwidth]{屏幕截图 2024-09-16 025357.png}
\item \textbf{使用反射动态设置类属性}:
\begin{lstlisting}
class MyClass:
    pass

setattr(MyClass, 'dynamic_attribute', 'dynamic_value')

obj = MyClass()
print(obj.dynamic_attribute)
\end{lstlisting}
动态设置类属性。\\
\includegraphics[width=0.5\textwidth]{屏幕截图 2024-09-16 025505.png}
\item \textbf{使用装饰器修改函数行为}:
\begin{lstlisting}
def add_logging(func):
    def wrapper(*args, **kwargs):
        print(f"Calling {func.__name__} with args: {args}, kwargs: {kwargs}")
        return func(*args, **kwargs)
    return wrapper

@add_logging
def greet(name):
    print(f"Hello, {name}!")

greet("Guolutong")
\end{lstlisting}
装饰器修改函数行为。\\
\includegraphics[width=0.5\textwidth]{屏幕截图 2024-09-16 025811.png}
\item \textbf{使用闭包记住外部变量}:
\begin{lstlisting}
def make_counter():
    count = [0]  # 使用列表来绕过局部变量不可变的问题
    def counter():
        count[0] += 1
        return count[0]
    return counter

counter = make_counter()
print(counter())
print(counter())
\end{lstlisting}
闭包记住外部变量。\\
\includegraphics[width=0.5\textwidth]{屏幕截图 2024-09-16 025855.png}
\item \textbf{使用getattr和setattr动态获取和设置属性}:
\begin{lstlisting}
class DynamicAttr:
    def __getattr__(self, name):
        return f"Value for {name}"

    def __setattr__(self, name, value):
        object.__setattr__(self, name, value)

obj = DynamicAttr()
print(obj.some_attribute)  
obj.some_attribute = 5
print(obj.some_attribute)  
\end{lstlisting}
\includegraphics[width=0.5\textwidth]{屏幕截图 2024-09-16 025959.png}
\item \textbf{使用exec执行字符串形式的代码}:
\begin{lstlisting}
code = """
def hello_world():
    print("Hello, World!")
"""
exec(code)
hello_world()
\end{lstlisting}
此时代码会报错,但这其实可以编译运行。\\
\includegraphics[width=0.5\textwidth]{屏幕截图 2024-09-16 030035.png}
\end{enumerate}

\section{PyTorch 代码示例}

以下是一些具体的 PyTorch 代码示例及其完整代码示例:

\begin{enumerate}[label=\arabic*.]
\item \textbf{创建张量}:
\begin{lstlisting}
import torch

x = torch.tensor([1.0, 2.0, 3.0])
print(x)
\end{lstlisting}
创建一个张量。\\
\includegraphics[width=0.5\textwidth]{屏幕截图 2024-09-16 030148.png}
\item \textbf{张量加法}:
\begin{lstlisting}
import torch

x = torch.tensor([1.0, 2.0, 3.0])
print(x)
y = torch.tensor([4.0, 5.0, 6.0])
z = x + y
print(z)
\end{lstlisting}
两个张量相加。\\
\includegraphics[width=0.5\textwidth]{屏幕截图 2024-09-16 030258.png}
\item \textbf{张量乘法}:
\begin{lstlisting}
import torch

x = torch.tensor([1.0, 2.0, 3.0])
y = torch.tensor([4.0, 5.0, 6.0])
w = torch.matmul(x, y)
print(w)

\end{lstlisting}
两个张量的矩阵乘法。\\
\includegraphics[width=0.5\textwidth]{屏幕截图 2024-09-16 030400.png}
\item \textbf{训练模型}:
\begin{lstlisting}
import torch
import torch.nn as nn
import torch.optim as optim
from torchvision import datasets, transforms
from torch.utils.data import DataLoader

# 数据预处理
transform = transforms.Compose([
    transforms.ToTensor(),
    transforms.Normalize((0.5,), (0.5,))
])

# 加载训练数据
trainset = datasets.MNIST(root='./data', train=True, download=True, transform=transform)
trainloader = DataLoader(trainset, batch_size=4, shuffle=True)


# 定义网络
class SimpleNet(nn.Module):
    def __init__(self):
        super(SimpleNet, self).__init__()
        self.fc1 = nn.Linear(784, 128)  # 784 is the size of the flattened image
        self.fc2 = nn.Linear(128, 10)

    def forward(self, x):
        x = x.view(x.size(0), -1)  # Flatten the image
        x = torch.relu(self.fc1(x))
        x = self.fc2(x)
        return x


model = SimpleNet()

# 损失函数和优化器
criterion = nn.CrossEntropyLoss()
optimizer = optim.SGD(model.parameters(), lr=0.01)

# 训练模型
for epoch in range(20):  # 迭代20个epoch
    for i, data in enumerate(trainloader, 0):
        inputs, labels = data
        optimizer.zero_grad()
        outputs = model(inputs)
        loss = criterion(outputs, labels)
        loss.backward()
        optimizer.step()

        if (i + 1) % 100 == 0:
            print(f'Epoch [{epoch + 1}/20], Step [{i + 1}/600], Loss: {loss.item():.4f}')

torch.save(model.state_dict(), 'output')
\end{lstlisting}
训练了一个手写数字识别的模型,下载了一个data,里面有raw,用于训练,自己创建一个.pth文件,用于保存模型,训练了一个可以预测手写数字的模型,这里训练20圈\\
\includegraphics[width=0.5\textwidth]{屏幕截图 2024-09-16 030543.png}
\item \textbf{使用模型}:
\begin{lstlisting}
import torch
from PIL import Image
import torchvision.transforms as transforms
import torch.nn as nn
class SimpleNet(nn.Module):
    def __init__(self):
        super(SimpleNet, self).__init__()
        self.fc1 = nn.Linear(784, 128)
        self.fc2 = nn.Linear(128, 10)

    def forward(self, x):
        x = x.view(x.size(0), -1)
        x = torch.relu(self.fc1(x))
        x = self.fc2(x)
        return x
# 加载模型
model = SimpleNet()
model_state_dict = torch.load('E:\pytorch_learn\output\\num.pth')
model.eval()

# 图像预处理
transform = transforms.Compose([
    transforms.Grayscale(),  # 转换为灰度图
    transforms.Resize((28, 28)),  # 调整图像大小为28x28
    transforms.ToTensor(),  # 转换为Tensor
    transforms.Normalize((0.5,), (0.5,))  # 归一化
])

# 加载图像
image_path = "E:\pytorch_learn\input\\6c004d86e8eab8dd654677cf1edd9b9.jpg"
image = Image.open(image_path).convert('L')  # 转换为灰度图

# 预处理图像
tensor = transform(image).unsqueeze(0)  # 增加一个批次维度

# 预测
with torch.no_grad():
    output = model(tensor)
    _, predicted = torch.max(output, 1)

# 显示结果
print(f'Predicted digit: {predicted.item()}')

\end{lstlisting}
使用该模型来预测手写数字,给它一张图片,他能预测数字,结果并不算理想\\
\includegraphics[width=0.5\textwidth]{6c004d86e8eab8dd654677cf1edd9b9.jpg}
\includegraphics[width=0.5\textwidth]{屏幕截图 2024-09-16 030948.png}
\item \textbf{使用 GPU}:
\begin{lstlisting}
import torch

# 检查是否有可用的CUDA设备
if torch.cuda.is_available():
    device = torch.device("cuda:0")
    print("Using CUDA")
else:
    device = torch.device("cpu")
    print("Using CPU")

# 将张量移动到GPU
tensor = torch.tensor([1.0, 2.0]).to(device)
print(tensor)
\end{lstlisting}
将模型和数据移到 GPU 上。

\item \textbf{自动求导}:
\begin{lstlisting}
import torch

# 创建一个张量,requires_grad设置为True
x = torch.tensor([1.0, 2.0, 3.0], requires_grad=True)

# 进行一些操作
y = x ** 2

# 计算梯度
y.backward(torch.tensor([1.0, 1.0, 1.0]))

# 打印梯度
print(x.grad)
\end{lstlisting}
输出该张量梯度\\
\includegraphics[width=0.5\textwidth]{屏幕截图 2024-09-16 031213.png}

\item \textbf{使用pytorch进行积分}:
\begin{lstlisting}
import torch

# 定义一个函数
def f(t):
    return t * torch.sin(t)

# 定义一个范围
t = torch.linspace(0., 5., 100)

# 近似积分
integral_approximation = torch.trapz(f(t), t)
print(integral_approximation)
\end{lstlisting}
用了梯形法则。\\
\includegraphics[width=0.5\textwidth]{屏幕截图 2024-09-16 031553.png}
\item \textbf{使用pytorch进行随机采样}:
\begin{lstlisting}
import torch

# 创建一个张量
x = torch.rand(5, 3)

# 随机选择一个元素
index = torch.randint(low=0, high=x.shape[0], size=(1,))
selected_element = x[index]
print(selected_element)
\end{lstlisting}
随机采样。\\
\includegraphics[width=0.5\textwidth]{屏幕截图 2024-09-16 031857.png}
\end{enumerate}

\section{感悟}
\begin{enumerate}
    \item 
通过学习的调试和性能分析技巧,我发现这些工具对于找出代码中的错误和提高程序效率非常有帮助。尤其是 ‘pdb’ 和 `cProfile`,可以让我找到问题出现的地方。
    \item 
元编程的概念让我意识到 Python 的灵活性,通过动态生成或修改类和函数,可以实现更高效的编程模式。例如,使用类装饰器可以轻松地为类添加新的功能,还有一些能够将字符串变成代码的操作,都让我眼前一亮。
    \item 
PyTorch 的使用让我感受到了深度学习的强大之处。从简单的张量操作到复杂的模型训练,PyTorch 提供了一套完整的工具链,使得训练变得更加容易,曾经我也训练过一些语言模型,对这个十分感兴趣,但用的都是别人现成的代码,这次是自己的代码,虽然过程十分曲折,但还是学习到很多新知识
    \item 
我希望在未来的学习和工作中,能够更加熟练地运用这些工具和技术,提高自己的工作效率。
\end{enumerate}

\section{GitHub 链接}
\url{https://github.com/gleati/work} 

\end{document}