\documentclass{ctexart}
\pagestyle{empty}
\usepackage[a4paper, margin=1in]{geometry}
\usepackage{listings}
\usepackage{xcolor}
\usepackage{enumitem}
\usepackage{hyperref}

% 设置代码块样式
\lstset{
  numbers=none,
  numberstyle=\tiny,
  numbersep=.1em,
  xleftmargin=0em,
  backgroundcolor=\color{lightgray!50},
  columns=fixed,
  breakindent = 1em,
  language=bash,
  breaklines=true,
  basicstyle=\ttfamily,
  tabsize=2,
}

\begin{document}

\title{Vim 命令和 Shell 脚本命令介绍}
\author{郭路通  23020007032}
\date{}
\maketitle

\section{Vim 命令}
以下是一些常用的 Vim 命令:

\begin{enumerate}[label=\arabic*.]
\item \textbf{打开文件}:
\begin{lstlisting}
vim <filename>
\end{lstlisting}
在 Vim 中打开一个文件。

\item \textbf{进入插入模式}:
\begin{lstlisting}
i
\end{lstlisting}
进入插入模式以编辑文本。

\item \textbf{保存并退出}:
\begin{lstlisting}
:wq
\end{lstlisting}
保存更改并退出 Vim。

\item \textbf{只退出}:
\begin{lstlisting}
:q
\end{lstlisting}
如果未修改,则直接退出 Vim;如果有未保存的更改,使用 :q! 强制退出。

\item \textbf{删除一行}:
\begin{lstlisting}
dd
\end{lstlisting}
删除光标所在行。

\item \textbf{撤销操作}:
\begin{lstlisting}
u
\end{lstlisting}
撤销上一步操作。

\item \textbf{重复操作}:
\begin{lstlisting}
.
\end{lstlisting}
重复最后一次更改。

\item \textbf{搜索文本}:
\begin{lstlisting}
/<pattern>
\end{lstlisting}
搜索指定模式的文本。

\item \textbf{替换文本}:
\begin{lstlisting}
:%s/old/new/g
\end{lstlisting}
全局替换文档中的文本(g 表示全局)。

\item \textbf{分割窗口}:
\begin{lstlisting}
:sp <filename>
\end{lstlisting}
在当前窗口旁边打开另一个文件。
\end{enumerate}

\section{Shell 脚本命令}
以下是一些复杂的 Shell 脚本示例:

\begin{enumerate}[label=\arabic*.]
\item \textbf{备份目录}:
\begin{lstlisting}
#!/bin/bash
SRC_DIR=$1
DEST_DIR=$2
BACKUP_DATE=$(date +"%Y%m%d_%H%M%S")
BACKUP_NAME="backup_$BACKUP_DATE.tar.gz"

tar -czf "$DEST_DIR/$BACKUP_NAME" -C "$SRC_DIR" .
echo "Backup created at $DEST_DIR/$BACKUP_NAME"
\end{lstlisting}

\item \textbf{检查文件是否为空}:
\begin{lstlisting}
#!/bin/bash
FILE=$1
if [ ! -s "$FILE" ]; then
    echo "The file $FILE is empty."
else
    echo "The file $FILE is not empty."
fi
\end{lstlisting}

\item \textbf{递归查找文件}:
\begin{lstlisting}
#!/bin/bash
DIR=$1
FILENAME=$2
find "$DIR" -type f -name "$FILENAME"
\end{lstlisting}

\item \textbf{计算目录大小}:
\begin{lstlisting}
#!/bin/bash
DIR=$1
du -sh "$DIR"
\end{lstlisting}

\item \textbf{生成随机密码}:
\begin{lstlisting}
#!/bin/bash
LENGTH=$1
cat /dev/urandom | tr -dc 'a-zA-Z0-9' | fold -w "$LENGTH" | head -n 1
\end{lstlisting}

\item \textbf{检查网络连接}:
\begin{lstlisting}
#!/bin/bash
HOST=$1
if ping -c 1 "$HOST" > /dev/null; then
    echo "Host $HOST is reachable."
else
    echo "Host $HOST is unreachable."
fi
\end{lstlisting}

\item \textbf{批量重命名文件}:
\begin{lstlisting}
#!/bin/bash
OLD_PREFIX=$1
NEW_PREFIX=$2
for FILE in "$OLD_PREFIX"*; do
    if [ -f "$FILE" ]; then
        NEW_NAME="${FILE//$OLD_PREFIX/$NEW_PREFIX}"
        mv "$FILE" "$NEW_NAME"
    fi
done
\end{lstlisting}

\item \textbf{压缩文件}:
\begin{lstlisting}
#!/bin/bash
FILE=$1
tar -czf "${FILE}.tar.gz" "$FILE"
echo "Compressed file ${FILE}.tar.gz"
\end{lstlisting}

\item \textbf{解压文件}:
\begin{lstlisting}
#!/bin/bash
FILE=$1
tar -xzf "$FILE"
echo "Extracted files from $FILE"
\end{lstlisting}

\item \textbf{定时任务}:
\begin{lstlisting}
#!/bin/bash
TASK="$1"
while true; do
    eval "$TASK"
    sleep 60  # 每分钟执行一次
done
\end{lstlisting}

\item \textbf{日志分析脚本}:
\begin{lstlisting}
#!/bin/bash
LOG_FILE=$1
grep "ERROR" "$LOG_FILE"
\end{lstlisting}

\item \textbf{文件比较}:
\begin{lstlisting}
#!/bin/bash
FILE1=$1
FILE2=$2
if cmp -s "$FILE1" "$FILE2"; then
    echo "Files are identical."
else
    echo "Files differ."
fi
\end{lstlisting}
\end{enumerate}

\section{感悟}
\begin{enumerate}
    \item 通过学习 Vim 命令,我认识到 Vim 是一个强大的文本编辑器,它允许我们以非常高效的方式进行文本编辑。熟练掌握 Vim 可以极大地提高编码速度和质量。
    \item 学习 Shell 脚本编写使我能够自动化日常任务,简化了数据处理和系统管理的工作流程。Shell 脚本是连接系统各个部分的强大工具。
    \item 这些技能的掌握不仅提高了个人效率,也加强了我在团队中的作用。我期待着将这些技能应用于实际项目中,解决更多的实际问题。
\end{enumerate}

\section{github链接}
\url{https://github.com/gleati/work}
\end{document}