\documentclass{ctexart}
\pagestyle{empty}
\usepackage[a4paper, margin=1in]{geometry}
\usepackage{graphicx}
\usepackage{listings}
\usepackage{xcolor}
\usepackage{enumitem}
\usepackage{hyperref}

% 设置代码块样式
\lstset{
  numbers=none,
  numberstyle=\tiny,
  numbersep=.1em,
  xleftmargin=0em,
  backgroundcolor=\color{lightgray!50},
  columns=fixed,
  breakindent = 1em,
  language=[LaTeX]TeX,
  breaklines=true,
  basicstyle=\ttfamily,
  tabsize=1,
}
\begin{document}

\title{命令行操作与 Python 编程实践报告}
\author{姓名:郭路通 \\ 学号:23020007032}
\date{}
\maketitle

\section{命令行环境操作}
以下是一些常用的命令:

\begin{enumerate}[label=\arabic*.]
\item \textbf{查看当前工作目录}:
\begin{lstlisting}
pwd
\end{lstlisting}
显示当前所在的工作目录。

\item \textbf{列出当前目录的内容}:
\begin{lstlisting}
ls
\end{lstlisting}
列出当前目录下的文件和目录。

\item \textbf{查看文件的前几行}:
\begin{lstlisting}
head filename
\end{lstlisting}
显示文件的前几行内容。

\item \textbf{查看文件的后几行}:
\begin{lstlisting}
tail filename
\end{lstlisting}
显示文件的后几行内容。

\item \textbf{搜索文件系统中的文件}:
\begin{lstlisting}
locate pattern
\end{lstlisting}
在文件系统中搜索符合模式的文件。
\end{enumerate}

\section{Python 简单代码}
以下是一些基本的 Python 代码示例:

\begin{enumerate}[label=\arabic*.]
\item \textbf{生成斐波那契数列}:
\begin{lstlisting}
def fibonacci(n):
    fib_sequence = [0, 1]
    while len(fib_sequence) < n:
        next_value = fib_sequence[-1] + fib_sequence[-2]
        fib_sequence.append(next_value)
    return fib_sequence[:n]

# 打印前20个斐波那契数
print(fibonacci(20))
\end{lstlisting}
\includegraphics[width=0.5\textwidth]{屏幕截图 2024-09-12 212753.png}
\item \textbf{实现阶乘函数(递归)}:
\begin{lstlisting}
def factorial(n):
    if n == 0:
        return 1
    else:
        return n * factorial(n-1)

# 计算 5 的阶乘
print(factorial(5))  # 输出应该是 120
\end{lstlisting}
\includegraphics[width=0.5\textwidth]{屏幕截图 2024-09-12 212743.png}
\item \textbf{猜数字游戏}:
\begin{lstlisting}
import random

def guess_number_game():
    target = random.randint(1, 100)
    guess = None
    
    while guess != target:
        guess = int(input("Guess a number between 1 and 100: "))
        if guess < target:
            print("小了!")
        elif guess > target:
            print("大了!")
    
    print("恭喜你,猜对了")

# 启动游戏
guess_number_game()
\end{lstlisting}
\includegraphics[width=0.5\textwidth]{屏幕截图 2024-09-12 212954.png}
\item \textbf{倒计时定时器}:
\begin{lstlisting}
import time


def countdown_timer(seconds):
    while seconds:
        mins, secs = divmod(seconds, 60)
        timer = '{:02d}:{:02d}'.format(mins, secs)
        print(timer, end="\r")
        time.sleep(1)
        seconds -= 1
        print(seconds)

    print("时间到!")


seconds = int(input("请输入倒计时秒数: "))
countdown_timer(seconds)
\end{lstlisting}
\includegraphics[width=0.5\textwidth]{屏幕截图 2024-09-12 213202.png}
\item \textbf{冒泡排序}:
\begin{lstlisting}
import random
import time


start = time.perf_counter()


a = []
for i in range(10):
    a.append(random.randint(1, 1000000))
print("随机生成的10个数字为")
print(a)
for k in range(9):
    for j in range(9 - k):
        if a[j] > a[j + 1]:
             a[j], a[j + 1] = a[j + 1], a[j]
print("排序之后的数组为")
print(a)
end = time.perf_counter()
print('运行时间:%s Seconds' % (end - start))


\end{lstlisting}
\includegraphics[width=0.5\textwidth]{屏幕截图 2024-09-12 213359.png}
\end{enumerate}

\section{Python 计算机视觉基础代码}
以下是一些基本的计算机视觉代码示例:

\begin{enumerate}[label=\arabic*.]
\item \textbf{读取图像并显示基本信息}:
\begin{lstlisting}
import cv2


def display_image_info(image_path):
    img = cv2.imread(image_path)
    if img is None:
        print(f"Error: 图像文件 {image_path} 不存在或无法读取.")
        return

    # 显示图像的基本信息
    print(f"图像尺寸: {img.shape}")
    print(f"图像数据类型: {img.dtype}")

    # 显示图像
    cv2.imshow('Image', img)
    cv2.waitKey(0)
    cv2.destroyAllWindows()


if __name__ == "__main__":
    image_path = "input/2024-08-20 223222.png"
    display_image_info(image_path)
\end{lstlisting}
\includegraphics[width=0.5\textwidth]{屏幕截图 2024-09-12 215556.png}
\item \textbf{转换颜色空间}:
\begin{lstlisting}
import cv2

def convert_color_space(image_path):
    img = cv2.imread(image_path)
    if img is None:
        print(f"Error: 图像文件 {image_path} 不存在或无法读取.")
        return
    
    # 将 BGR 图像转换为 HSV
    hsv_image = cv2.cvtColor(img, cv2.COLOR_BGR2HSV)
    
    # 将 BGR 图像转换为 RGB
    rgb_image = cv2.cvtColor(img, cv2.COLOR_BGR2RGB)
    
    # 显示原始图像和转换后的图像
    cv2.imshow('Original Image', img)
    cv2.imshow('HSV Image', hsv_image)
    cv2.imshow('RGB Image', rgb_image)
    
    cv2.waitKey(0)
    cv2.destroyAllWindows()

if __name__ == "__main__":
    image_path = r'路径/到/你的/图像.jpg'
    convert_color_space(image_path)
\end{lstlisting}
\includegraphics[width=0.5\textwidth]{屏幕截图 2024-09-12 220843.png}
\includegraphics[width=0.5\textwidth]{屏幕截图 2024-09-12 220851.png}
\item \textbf{图像缩放}:
\begin{lstlisting}
import cv2


def resize_image(image_path):
    img = cv2.imread(image_path)
    if img is None:
        print(f"Error: 图像文件 {image_path} 不存在或无法读取.")
        return

    # 缩放图像
    scale_percent = 50  # 缩放比例
    width = int(img.shape[1] * scale_percent / 100)
    height = int(img.shape[0] * scale_percent / 100)
    dim = (width, height)
    resized = cv2.resize(img, dim, interpolation=cv2.INTER_AREA)

    # 显示原始图像和缩放后的图像
    cv2.imshow('Original Image', img)
    cv2.imshow('Resized Image', resized)

    cv2.waitKey(0)
    cv2.destroyAllWindows()


if __name__ == "__main__":
    image_path = "input/2024-08-20 223222.png"
    resize_image(image_path)
\end{lstlisting}
\includegraphics[width=0.5\textwidth]{屏幕截图 2024-09-12 221058.png}
\item \textbf{图像裁剪}:
\begin{lstlisting}
import cv2


def crop_image(image_path):
    img = cv2.imread(image_path)
    if img is None:
        print(f"Error: 图像文件 {image_path} 不存在或无法读取.")
        return

    # 定义裁剪区域
    cropped = img[50:200, 100:300]

    # 显示原始图像和裁剪后的图像
    cv2.imshow('Original Image', img)
    cv2.imshow('Cropped Image', cropped)

    cv2.waitKey(0)
    cv2.destroyAllWindows()


if __name__ == "__main__":
    image_path = "input/2024-08-20 223222.png"
    crop_image(image_path)
\end{lstlisting}
\includegraphics[width=0.5\textwidth]{屏幕截图 2024-09-12 221435.png}
\item \textbf{高斯模糊}:
\begin{lstlisting}
import cv2


def apply_gaussian_blur(image_path):
    img = cv2.imread(image_path)
    if img is None:
        print(f"Error: 图像文件 {image_path} 不存在或无法读取.")
        return

    # 应用高斯模糊
    blurred = cv2.GaussianBlur(img, (5, 5), 0)

    # 显示原始图像和模糊后的图像
    cv2.imshow('Original Image', img)
    cv2.imshow('Gaussian Blurred Image', blurred)

    cv2.waitKey(0)
    cv2.destroyAllWindows()


if __name__ == "__main__":
    image_path = "input/2024-08-20 223222.png"
    apply_gaussian_blur(image_path)
\end{lstlisting}
\includegraphics[width=0.5\textwidth]{屏幕截图 2024-09-12 221746.png}
\item \textbf{Sobel 边缘检测}:
\begin{lstlisting}
import cv2
import numpy as np


def sobel_edge_detection(image_path):
    img = cv2.imread(image_path)
    if img is None:
        print(f"Error: 图像文件 {image_path} 不存在或无法读取.")
        return

    # 转换为灰度图像
    gray_image = cv2.cvtColor(img, cv2.COLOR_BGR2GRAY)

    # 应用 Sobel 操作算子
    sobelx = cv2.Sobel(gray_image, cv2.CV_64F, 1, 0, ksize=5)
    sobely = cv2.Sobel(gray_image, cv2.CV_64F, 0, 1, ksize=5)

    # 显示原始图像和 Sobel 边缘检测结果
    cv2.imshow('Original Image', img)
    cv2.imshow('Sobel X', sobelx)
    cv2.imshow('Sobel Y', sobely)

    cv2.waitKey(0)
    cv2.destroyAllWindows()


if __name__ == "__main__":
    image_path = "input/2024-08-20 223222.png"
    sobel_edge_detection(image_path)
\end{lstlisting}
\includegraphics[width=0.5\textwidth]{屏幕截图 2024-09-12 221842.png}
\item \textbf{灰度}:
\begin{lstlisting}
import cv2
import numpy as np

def main():
    # 读取图像
    image_path = 'input/2024-08-20 223222.png'
    img = cv2.imread(image_path)

    if img is None:
        print("Error: 图像文件不存在或无法读取.")
        return
    # 转换为灰度图像
    gray_image = cv2.cvtColor(img, cv2.COLOR_BGR2GRAY)

    # 使用 Canny 算法检测边缘
    edges = cv2.Canny(gray_image, threshold1=50, threshold2=150)

    # 显示原图和处理后的图像
    cv2.imshow('Original Image', img)
    cv2.imshow('Gray Image', gray_image)
    cv2.imshow('Edges', edges)

    # 等待按键后关闭窗口
    cv2.waitKey(0)
    cv2.destroyAllWindows()

if __name__ == "__main__":
    main()
\end{lstlisting}
\includegraphics[width=0.5\textwidth]{屏幕截图 2024-09-12 222008.png}
\includegraphics[width=0.5\textwidth]{屏幕截图 2024-09-12 222103.png}
\item \textbf{PNG 格式的图像转换为 JPG 格式}:
\begin{lstlisting}
import cv2


def convert_image_format(input_path, output_path):
    """
    读取 PNG 格式的图像,并将其转换为 JPEG 格式。

    :param input_path: 输入 PNG 图像的文件路径
    :param output_path: 输出 JPEG 图像的文件路径
    """
    # 读取图像
    img = cv2.imread(input_path, cv2.IMREAD_UNCHANGED)

    # 检查是否成功读取图像
    if img is None:
        print(f"Error: 图像文件 {input_path} 不存在或无法读取.")
        return

    # 保存转换后的图像为 JPEG 格式
    cv2.imwrite(output_path, img)

    print(f"Image has been saved to {output_path}")


if __name__ == "__main__":
    # 提供一个 PNG 图像文件的路径
    input_path = 'input/2024-08-20 223222.png'

    # 设置输出路径
    output_path = 'output/a.jpg'

    # 调用函数转换图像格式
    convert_image_format(input_path, output_path)
\end{lstlisting}
\includegraphics[width=0.5\textwidth]{屏幕截图 2024-09-12 223035.png}
\includegraphics[width=0.5\textwidth]{屏幕截图 2024-09-12 223041.png}
\item \textbf{旋转图片}:
\begin{lstlisting}
import cv2
import numpy as np


def rotate_image(image_path, angle):

    # 读取图像
    img = cv2.imread(image_path)

    # 检查是否成功读取图像
    if img is None:
        print(f"Error: 图像文件 {image_path} 不存在或无法读取.")
        return

    # 获取图像的高度和宽度
    (height, width) = img.shape[:2]

    # 计算图像中心
    center = (width // 2, height // 2)

    # 构建旋转矩阵
    rotation_matrix = cv2.getRotationMatrix2D(center, angle, 1.0)

    # 应用仿射变换
    rotated = cv2.warpAffine(img, rotation_matrix, (width, height))

    # 显示旋转后的图像
    cv2.imshow("Original Image", img)
    cv2.imshow("Rotated Image", rotated)

    # 等待用户按键,按任意键则关闭窗口
    cv2.waitKey(0)

    # 清理所有打开的窗口
    cv2.destroyAllWindows()


if __name__ == "__main__":
    # 提供一个图像文件的路径
    image_path = "input/2024-08-20 223222.png"

    # 设置旋转的角度
    angle = 45  # 例如旋转 45 度

    # 调用函数旋转并显示图像
    rotate_image(image_path, angle)
\end{lstlisting}
\includegraphics[width=0.5\textwidth]{屏幕截图 2024-09-12 223328.png}
\item \textbf{直方图均衡化}:
\begin{lstlisting}
import cv2
import numpy as np


def histogram_equalization(image_path):
    # 读取图像
    img = cv2.imread(image_path, cv2.IMREAD_GRAYSCALE)
    if img is None:
        print(f"Error: 图像文件 {image_path} 不存在或无法读取.")
        return

    # 直方图均衡化
    eq_img = cv2.equalizeHist(img)

    # 显示原始图像和均衡化后的图像
    cv2.imshow('Original Image', img)
    cv2.imshow('Equalized Image', eq_img)

    # 等待按键并关闭窗口
    cv2.waitKey(0)
    cv2.destroyAllWindows()


if __name__ == "__main__":
    image_path = "input/2024-08-20 223222.png"
    histogram_equalization(image_path)
\end{lstlisting}
\includegraphics[width=0.5\textwidth]{屏幕截图 2024-09-12 223515.png}
\includegraphics[width=0.5\textwidth]{屏幕截图 2024-09-12 223525.png}
\end{enumerate}
\section{感悟}
\begin{enumerate}
    \item 通过学习和使用 Python 的基础知识,我深刻体会到了这门语言的强大和灵活性。Python 的简洁语法和丰富的标准库使得编程变得更加直观和高效。无论是简单的数据处理还是复杂的算法实现,Python 都能提供强大的支持。例如,使用列表推导式和字典推导式可以让代码更加紧凑,而内置的函数则使得数据处理变得更加灵活。通过这些实践,我不仅提高了编程效率,也加深了对编程原理的理解。
    
    \item 在学习 Python 的计算机视觉基础时,我发现 OpenCV 是一个功能强大且易于使用的库。通过使用 OpenCV 进行图像读取、显示和处理,我能够直观地看到图像处理的效果。这对于理解图像的基本属性和格式非常有帮助。学习了如何使用 OpenCV 进行图像缩放、旋转、裁剪等基本操作,这些技能在处理图像数据集时非常有用。同时,掌握了如何使用 Canny 边缘检测算法和 SIFT 特征提取算法,使我能够识别图像中的关键特征,这对于计算机视觉任务如目标检测和图像分类非常重要。
    
    \item 这些工具的掌握,不仅提升了我的技术能力,也增强了我的责任感和团队合作精神。它们教会了我如何更有效地管理我的工作,无论是代码还是文档,都能够保持整洁和有序。在未来的学习和工作中,我期待继续深化这些技能,并将其应用到更广泛的领域中。
\end{enumerate}

\section{GitHub 链接}
\url{https://github.com/gleati/work}
\end{document}