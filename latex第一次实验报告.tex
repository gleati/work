\documentclass{ctexart}
\pagestyle{empty}
\usepackage[a4paper, margin=1in]{geometry}
\usepackage{listings}
\usepackage{xcolor}
\usepackage{enumitem}
\usepackage{hyperref}

% 设置代码块样式
\lstset{
  numbers=none,
  numberstyle=\tiny,
  numbersep=.1em,
  xleftmargin=0em,
  backgroundcolor=\color{lightgray!50},
  columns=fixed,
  breakindent = 1em,
  language=[LaTeX]TeX,
  breaklines=true,
  basicstyle=\ttfamily,
  tabsize=1,
}

\begin{document}

\title{Git 命令和 LaTeX 语法介绍}
\author{郭路通  23020007032}
\date{}
\maketitle

\section{Git 命令}
以下是一些常用的 Git 命令:

\begin{enumerate}[label=\arabic*.]
\item \textbf{初始化仓库}:
\begin{lstlisting}
git init
\end{lstlisting}
初始化一个新的 Git 仓库。

\item \textbf{添加文件}:
\begin{lstlisting}
git add <file>
\end{lstlisting}
将指定文件添加到暂存区。

\item \textbf{提交更改}:
\begin{lstlisting}
git commit -m "<message>"
\end{lstlisting}
提交暂存区文件到仓库,并附上说明信息。

\item \textbf{查看状态}:
\begin{lstlisting}
git status
\end{lstlisting}
显示当前仓库的状态。

\item \textbf{查看日志}:
\begin{lstlisting}
git log
\end{lstlisting}
显示提交日志。

\item \textbf{查看分支}:
\begin{lstlisting}
git branch
\end{lstlisting}
列出所有分支。

\item \textbf{切换分支}:
\begin{lstlisting}
git checkout <branch>
\end{lstlisting}
切换到指定分支。

\item \textbf{合并分支}:
\begin{lstlisting}
git merge <branch>
\end{lstlisting}
将指定分支的更改合并到当前分支。

\item \textbf{克隆仓库}:
\begin{lstlisting}
git clone <repository>
\end{lstlisting}
克隆远程仓库到本地。

\item \textbf{推送更改}:
\begin{lstlisting}
git push
\end{lstlisting}
推送本地仓库到远程仓库。

\item \textbf{拉取更改}:
\begin{lstlisting}
git pull
\end{lstlisting}
从远程仓库拉取最新内容。
\end{enumerate}

\section{LaTeX 语法}
以下是一些基本的 LaTeX 语法:

\begin{enumerate}[label=\arabic*.]
\item \textbf{文档类声明}:
\begin{lstlisting}
\documentclass{article}
\end{lstlisting}
定义文档类型。

\item \textbf{导入宏包}:
\begin{lstlisting}
\usepackage{graphicx}
\end{lstlisting}
导入额外的宏包。

\item \textbf{文档开始}:
\begin{lstlisting}
\begin{document}
\end{lstlisting}
标记文档内容的开始。

\item \textbf{设置标题}:
\begin{lstlisting}
\title{Title}
\end{lstlisting}
定义文档标题。

\item \textbf{设置作者}:
\begin{lstlisting}
\author{Author}
\end{lstlisting}
定义文档作者。

\item \textbf{设置日期}:
\begin{lstlisting}
\date{\today}
\end{lstlisting}
定义文档日期。

\item \textbf{生成标题}:
\begin{lstlisting}
\maketitle
\end{lstlisting}
生成文档的标题。

\item \textbf{节标题}:
\begin{lstlisting}
\section{Introduction}
\end{lstlisting}
创建新的节。

\item \textbf{无序列表}:
\begin{lstlisting}
\begin{itemize}
  \item Item 1
  \item Item 2
\end{itemize}
\end{lstlisting}
创建无序列表。

\item \textbf{有序列表}:
\begin{lstlisting}
\begin{enumerate}
  \item First item
  \item Second item
\end{enumerate}
\end{lstlisting}
创建有序列表。

\item \textbf{居中文本}:
\begin{lstlisting}
\begin{center}
  Centered text
\end{center}
\end{lstlisting}
将文本居中显示。

\item \textbf{创建表格}:
\begin{lstlisting}
\begin{tabular}{ll}
  Column1 & Column2 \\
  Data1   & Data2
\end{tabular}
\end{lstlisting}
创建一个简单的表格。

\item \textbf{插入图片}:
\begin{lstlisting}
\includegraphics{image.png}
\end{lstlisting}
在文档中插入图片。
\end{enumerate}

\section{感悟}
\begin{enumerate}
    \item 
通过学习和使用 Git 命令,我深刻体会到了版本控制对于软件开发和文档管理的重要性。Git 不仅是一个工具,更是现代协作开发的核心。它使得代码的提交、分支的管理和远程仓库的同步变得简单而高效。每一个命令,如 `git commit`、`git push` 和 `git pull`,都是团队协作中不可或缺的一部分,它们共同构成了项目开发的骨架。
    \item 
同时,LaTeX 语法的学习让我认识到了高质量文档排版的可能性。LaTeX 的强大之处在于其精确和灵活的排版系统,它能够生成从简单的文章到复杂的科学论文的各种文档。每一个环境和命令,如 section、begin{enumerate} 和 includegraphics,都是构建专业文档的基石。通过 LaTeX,我学会了如何将注意力集中在内容创作上,而不必担心格式和布局的问题。
    \item 
这些工具的掌握,不仅提升了我的技术能力,也增强了我的责任感和团队合作精神。它们教会了我如何更有效地管理我的工作,无论是代码还是文档,都能够保持整洁和有序。在未来的学习和工作中,我期待继续深化这些技能,并将其应用到更广泛的领域中。
\end{enumerate}

\section{github链接}
\url{https://github.com/gleati/work}
\end{document}
